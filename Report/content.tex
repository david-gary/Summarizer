\setlength{\headheight}{15pt}
\lhead{\textbf{ITCS-6150}}
\chead{\textbf{Group 7}}
\rhead{\textbf{Final Report}}

\vspace{-2pt}

\begin{center}
    \begin{Large}
        \noindent\textbf{A Survey on the Limitations of Text Summarization Models}
    \end{Large}
\end{center}

\vspace{-10pt}

% Two column display for Team Members and Models & Datasets
\begin{multicols}{2}

    \begin{center}
        \begin{large}
            \noindent\textbf{Team Members}
        \end{large}
    \end{center}
    % Table of group members and their ids
    % Correct column widths for a table
    % 2 columns, 2 rows, no header
    \begin{center}
        \begin{tabular}{|lc|}
            \hline
            \textbf{David Gary} & \href{mailto:dgary9@uncc.edu}{801325583} \\ \hline
            \textbf{Michael Smalenberger} & \href{mailto:msmalenb@uncc.edu}{800984973} \\ \hline
            \textbf{Matthew Seman} & \href{mailto:mseman1@uncc.edu}{801156143} \\ \hline
            \textbf{Ryan Marsh} & \href{mailto:rmarsh4@uncc.edu}{800552800} \\ \hline
        \end{tabular}
    \end{center}

    \begin{center}
        \begin{large}
            \noindent\textbf{Models}
        \end{large}
    \end{center}

    \begin{center}
        \begin{tabular}{|lc|}
            \hline
            \href{https://github.com/zihangdai/xlnet}{XLNet Model} & \href{https://huggingface.co/t5-base}{T5 Model} \\ \hline
            \href{https://huggingface.co/facebook/bart-large-cnn}{BART Model} &  \href{https://huggingface.co/facebook/bart-large-xsum}{BART-X Model} \\ \hline
            \href{https://huggingface.co/google/pegasus-large}{Pegasus Model} &  \href{https://huggingface.co/google/pegasus-xsum}{Pegasus-X Model} \\ \hline
        \end{tabular}
    \end{center}


    \begin{center}
        \begin{large}
            \noindent\textbf{Sentiment Analysis Datasets}
        \end{large}
    \end{center}

    \begin{center}
        \begin{tabular}{|c|}
            \hline
            \href{https://paperswithcode.com/dataset/reddit-tifu}{Reddit TIFU Dataset} \\ \hline
            \href{https://paperswithcode.com/dataset/multi-news}{Multi-News Dataset} \\ \hline
            \href{https://paperswithcode.com/dataset/s2orc}{S2ORC Dataset} \\ \hline
        \end{tabular}
    \end{center}

    \begin{center}
        \begin{large}
            \noindent\textbf{Summarization Datasets}
        \end{large}
    \end{center}

    \begin{center}
        \begin{tabular}{|c|}
            \hline
            \href{https://huggingface.co/datasets/gigaword}{Gigaword Dataset} \\ \hline
            \href{https://huggingface.co/datasets/cnn_dailymail}{CNN/DailyMail Dataset} \\ \hline
            \href{https://huggingface.co/datasets/xsum}{XSum Dataset} \\ \hline
        \end{tabular}
    \end{center}

\end{multicols}

\begin{large}
    \noindent\textbf{Introduction}
\end{large}

\vspace{5pt}

Transformers allow us to vectorize text and analyze the relationships between words and sentences.
This technique allows for more powerful and larger language models to be created.
Accuracy assessments through expert review are often unavailable for highly-contextual text summarization, so standardized evaluation schemes like Pyramid or ROUGE are used.
A growing body of literature focuses on assessing these metrics' efficacy and limitations.
Our goal is to review the literature and summarize the findings of these studies.
In addition, we aim to apply the XLNet transformer model to various datasets from distinct contexts and thoroughly review the cost and limitations of generalizable text summarization.

\vspace{2pt}

Our proposed deliverables and the subsequent system will perform text analysis and summarization, following the current standards of generalizability.
Anyone can use this system to summarize or assess the tone of text input.
Since the point here is to show the limitations of generalized text summarization models, perfect accuracy is not the goal of our system.
Users should beware of this and, as a result, only utilize this tool for interpretation guidance.

\vspace{5pt}

\begin{large}
    \noindent\textbf{Deliverable 1: Length-Agnostic Sentiment Analysis}
\end{large}

Our first deliverable will have a foundational ability to assess the general sentiment of an input text.
We will utilize the XLNet transformer model as a backbone and employ some of the length-agnostic methods shown in the Guan, Smetannikov, and Tianxing paper\cite{Transformers}. 

\vspace{1pt}

\noindent\textbf{Use Cases:}

\vspace{0pt}

% enumerate with small text and small spacing
% text size should be smaller than the rest of the document
\begin{enumerate}[label=\textbf{\arabic*}, labelsep=0.1em, itemsep=0em, topsep=0em, font=\small]
    \item Automate customer feedback interpretation from collected surveys.
    \item Assessing the emotional intensity of messages without context.
    \item Mining general public opinion on a specific topic from public forums such as Reddit.
    \item Score attempts at persuading individuals after targeted advertising.
\end{enumerate}

\vspace{5pt}

\begin{large}
    \noindent\textbf{Deliverable 2: Text Summarizer}
\end{large}

Next, we will build on the XLNet framework to produce a generalized text summarizer.
This will include a score report that shows how well the model performs according to the Pyramid and ROUGE schemes.

\vspace{1pt}

\noindent\textbf{Use Cases:}

\vspace{0pt}

\begin{enumerate}[label=\textbf{\arabic*}, labelsep=0.1em, itemsep=0em, topsep=0em, font=\small]
    \item Create a highlights reel of current news articles.
    \item Automatically summarize chapters from a book.
    \item Distill user messages for chatbots to understand more easily.
    \item Extract key information from a medical patient’s electronic health records.
\end{enumerate}